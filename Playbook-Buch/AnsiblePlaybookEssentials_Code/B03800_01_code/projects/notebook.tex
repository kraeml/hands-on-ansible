
% Default to the notebook output style

    


% Inherit from the specified cell style.




    
\documentclass[11pt]{article}

    
    
    \usepackage[T1]{fontenc}
    % Nicer default font (+ math font) than Computer Modern for most use cases
    \usepackage{mathpazo}

    % Basic figure setup, for now with no caption control since it's done
    % automatically by Pandoc (which extracts ![](path) syntax from Markdown).
    \usepackage{graphicx}
    % We will generate all images so they have a width \maxwidth. This means
    % that they will get their normal width if they fit onto the page, but
    % are scaled down if they would overflow the margins.
    \makeatletter
    \def\maxwidth{\ifdim\Gin@nat@width>\linewidth\linewidth
    \else\Gin@nat@width\fi}
    \makeatother
    \let\Oldincludegraphics\includegraphics
    % Set max figure width to be 80% of text width, for now hardcoded.
    \renewcommand{\includegraphics}[1]{\Oldincludegraphics[width=.8\maxwidth]{#1}}
    % Ensure that by default, figures have no caption (until we provide a
    % proper Figure object with a Caption API and a way to capture that
    % in the conversion process - todo).
    \usepackage{caption}
    \DeclareCaptionLabelFormat{nolabel}{}
    \captionsetup{labelformat=nolabel}

    \usepackage{adjustbox} % Used to constrain images to a maximum size 
    \usepackage{xcolor} % Allow colors to be defined
    \usepackage{enumerate} % Needed for markdown enumerations to work
    \usepackage{geometry} % Used to adjust the document margins
    \usepackage{amsmath} % Equations
    \usepackage{amssymb} % Equations
    \usepackage{textcomp} % defines textquotesingle
    % Hack from http://tex.stackexchange.com/a/47451/13684:
    \AtBeginDocument{%
        \def\PYZsq{\textquotesingle}% Upright quotes in Pygmentized code
    }
    \usepackage{upquote} % Upright quotes for verbatim code
    \usepackage{eurosym} % defines \euro
    \usepackage[mathletters]{ucs} % Extended unicode (utf-8) support
    \usepackage[utf8x]{inputenc} % Allow utf-8 characters in the tex document
    \usepackage{fancyvrb} % verbatim replacement that allows latex
    \usepackage{grffile} % extends the file name processing of package graphics 
                         % to support a larger range 
    % The hyperref package gives us a pdf with properly built
    % internal navigation ('pdf bookmarks' for the table of contents,
    % internal cross-reference links, web links for URLs, etc.)
    \usepackage{hyperref}
    \usepackage{longtable} % longtable support required by pandoc >1.10
    \usepackage{booktabs}  % table support for pandoc > 1.12.2
    \usepackage[inline]{enumitem} % IRkernel/repr support (it uses the enumerate* environment)
    \usepackage[normalem]{ulem} % ulem is needed to support strikethroughs (\sout)
                                % normalem makes italics be italics, not underlines
    

    
    
    % Colors for the hyperref package
    \definecolor{urlcolor}{rgb}{0,.145,.698}
    \definecolor{linkcolor}{rgb}{.71,0.21,0.01}
    \definecolor{citecolor}{rgb}{.12,.54,.11}

    % ANSI colors
    \definecolor{ansi-black}{HTML}{3E424D}
    \definecolor{ansi-black-intense}{HTML}{282C36}
    \definecolor{ansi-red}{HTML}{E75C58}
    \definecolor{ansi-red-intense}{HTML}{B22B31}
    \definecolor{ansi-green}{HTML}{00A250}
    \definecolor{ansi-green-intense}{HTML}{007427}
    \definecolor{ansi-yellow}{HTML}{DDB62B}
    \definecolor{ansi-yellow-intense}{HTML}{B27D12}
    \definecolor{ansi-blue}{HTML}{208FFB}
    \definecolor{ansi-blue-intense}{HTML}{0065CA}
    \definecolor{ansi-magenta}{HTML}{D160C4}
    \definecolor{ansi-magenta-intense}{HTML}{A03196}
    \definecolor{ansi-cyan}{HTML}{60C6C8}
    \definecolor{ansi-cyan-intense}{HTML}{258F8F}
    \definecolor{ansi-white}{HTML}{C5C1B4}
    \definecolor{ansi-white-intense}{HTML}{A1A6B2}

    % commands and environments needed by pandoc snippets
    % extracted from the output of `pandoc -s`
    \providecommand{\tightlist}{%
      \setlength{\itemsep}{0pt}\setlength{\parskip}{0pt}}
    \DefineVerbatimEnvironment{Highlighting}{Verbatim}{commandchars=\\\{\}}
    % Add ',fontsize=\small' for more characters per line
    \newenvironment{Shaded}{}{}
    \newcommand{\KeywordTok}[1]{\textcolor[rgb]{0.00,0.44,0.13}{\textbf{{#1}}}}
    \newcommand{\DataTypeTok}[1]{\textcolor[rgb]{0.56,0.13,0.00}{{#1}}}
    \newcommand{\DecValTok}[1]{\textcolor[rgb]{0.25,0.63,0.44}{{#1}}}
    \newcommand{\BaseNTok}[1]{\textcolor[rgb]{0.25,0.63,0.44}{{#1}}}
    \newcommand{\FloatTok}[1]{\textcolor[rgb]{0.25,0.63,0.44}{{#1}}}
    \newcommand{\CharTok}[1]{\textcolor[rgb]{0.25,0.44,0.63}{{#1}}}
    \newcommand{\StringTok}[1]{\textcolor[rgb]{0.25,0.44,0.63}{{#1}}}
    \newcommand{\CommentTok}[1]{\textcolor[rgb]{0.38,0.63,0.69}{\textit{{#1}}}}
    \newcommand{\OtherTok}[1]{\textcolor[rgb]{0.00,0.44,0.13}{{#1}}}
    \newcommand{\AlertTok}[1]{\textcolor[rgb]{1.00,0.00,0.00}{\textbf{{#1}}}}
    \newcommand{\FunctionTok}[1]{\textcolor[rgb]{0.02,0.16,0.49}{{#1}}}
    \newcommand{\RegionMarkerTok}[1]{{#1}}
    \newcommand{\ErrorTok}[1]{\textcolor[rgb]{1.00,0.00,0.00}{\textbf{{#1}}}}
    \newcommand{\NormalTok}[1]{{#1}}
    
    % Additional commands for more recent versions of Pandoc
    \newcommand{\ConstantTok}[1]{\textcolor[rgb]{0.53,0.00,0.00}{{#1}}}
    \newcommand{\SpecialCharTok}[1]{\textcolor[rgb]{0.25,0.44,0.63}{{#1}}}
    \newcommand{\VerbatimStringTok}[1]{\textcolor[rgb]{0.25,0.44,0.63}{{#1}}}
    \newcommand{\SpecialStringTok}[1]{\textcolor[rgb]{0.73,0.40,0.53}{{#1}}}
    \newcommand{\ImportTok}[1]{{#1}}
    \newcommand{\DocumentationTok}[1]{\textcolor[rgb]{0.73,0.13,0.13}{\textit{{#1}}}}
    \newcommand{\AnnotationTok}[1]{\textcolor[rgb]{0.38,0.63,0.69}{\textbf{\textit{{#1}}}}}
    \newcommand{\CommentVarTok}[1]{\textcolor[rgb]{0.38,0.63,0.69}{\textbf{\textit{{#1}}}}}
    \newcommand{\VariableTok}[1]{\textcolor[rgb]{0.10,0.09,0.49}{{#1}}}
    \newcommand{\ControlFlowTok}[1]{\textcolor[rgb]{0.00,0.44,0.13}{\textbf{{#1}}}}
    \newcommand{\OperatorTok}[1]{\textcolor[rgb]{0.40,0.40,0.40}{{#1}}}
    \newcommand{\BuiltInTok}[1]{{#1}}
    \newcommand{\ExtensionTok}[1]{{#1}}
    \newcommand{\PreprocessorTok}[1]{\textcolor[rgb]{0.74,0.48,0.00}{{#1}}}
    \newcommand{\AttributeTok}[1]{\textcolor[rgb]{0.49,0.56,0.16}{{#1}}}
    \newcommand{\InformationTok}[1]{\textcolor[rgb]{0.38,0.63,0.69}{\textbf{\textit{{#1}}}}}
    \newcommand{\WarningTok}[1]{\textcolor[rgb]{0.38,0.63,0.69}{\textbf{\textit{{#1}}}}}
    
    
    % Define a nice break command that doesn't care if a line doesn't already
    % exist.
    \def\br{\hspace*{\fill} \\* }
    % Math Jax compatability definitions
    \def\gt{>}
    \def\lt{<}
    % Document parameters
    \title{01\_Playbooks}
    
    
    

    % Pygments definitions
    
\makeatletter
\def\PY@reset{\let\PY@it=\relax \let\PY@bf=\relax%
    \let\PY@ul=\relax \let\PY@tc=\relax%
    \let\PY@bc=\relax \let\PY@ff=\relax}
\def\PY@tok#1{\csname PY@tok@#1\endcsname}
\def\PY@toks#1+{\ifx\relax#1\empty\else%
    \PY@tok{#1}\expandafter\PY@toks\fi}
\def\PY@do#1{\PY@bc{\PY@tc{\PY@ul{%
    \PY@it{\PY@bf{\PY@ff{#1}}}}}}}
\def\PY#1#2{\PY@reset\PY@toks#1+\relax+\PY@do{#2}}

\expandafter\def\csname PY@tok@gt\endcsname{\def\PY@tc##1{\textcolor[rgb]{0.00,0.27,0.87}{##1}}}
\expandafter\def\csname PY@tok@nv\endcsname{\def\PY@tc##1{\textcolor[rgb]{0.10,0.09,0.49}{##1}}}
\expandafter\def\csname PY@tok@fm\endcsname{\def\PY@tc##1{\textcolor[rgb]{0.00,0.00,1.00}{##1}}}
\expandafter\def\csname PY@tok@ow\endcsname{\let\PY@bf=\textbf\def\PY@tc##1{\textcolor[rgb]{0.67,0.13,1.00}{##1}}}
\expandafter\def\csname PY@tok@se\endcsname{\let\PY@bf=\textbf\def\PY@tc##1{\textcolor[rgb]{0.73,0.40,0.13}{##1}}}
\expandafter\def\csname PY@tok@sa\endcsname{\def\PY@tc##1{\textcolor[rgb]{0.73,0.13,0.13}{##1}}}
\expandafter\def\csname PY@tok@w\endcsname{\def\PY@tc##1{\textcolor[rgb]{0.73,0.73,0.73}{##1}}}
\expandafter\def\csname PY@tok@kt\endcsname{\def\PY@tc##1{\textcolor[rgb]{0.69,0.00,0.25}{##1}}}
\expandafter\def\csname PY@tok@mh\endcsname{\def\PY@tc##1{\textcolor[rgb]{0.40,0.40,0.40}{##1}}}
\expandafter\def\csname PY@tok@kp\endcsname{\def\PY@tc##1{\textcolor[rgb]{0.00,0.50,0.00}{##1}}}
\expandafter\def\csname PY@tok@vg\endcsname{\def\PY@tc##1{\textcolor[rgb]{0.10,0.09,0.49}{##1}}}
\expandafter\def\csname PY@tok@s\endcsname{\def\PY@tc##1{\textcolor[rgb]{0.73,0.13,0.13}{##1}}}
\expandafter\def\csname PY@tok@mf\endcsname{\def\PY@tc##1{\textcolor[rgb]{0.40,0.40,0.40}{##1}}}
\expandafter\def\csname PY@tok@s2\endcsname{\def\PY@tc##1{\textcolor[rgb]{0.73,0.13,0.13}{##1}}}
\expandafter\def\csname PY@tok@c1\endcsname{\let\PY@it=\textit\def\PY@tc##1{\textcolor[rgb]{0.25,0.50,0.50}{##1}}}
\expandafter\def\csname PY@tok@nt\endcsname{\let\PY@bf=\textbf\def\PY@tc##1{\textcolor[rgb]{0.00,0.50,0.00}{##1}}}
\expandafter\def\csname PY@tok@go\endcsname{\def\PY@tc##1{\textcolor[rgb]{0.53,0.53,0.53}{##1}}}
\expandafter\def\csname PY@tok@s1\endcsname{\def\PY@tc##1{\textcolor[rgb]{0.73,0.13,0.13}{##1}}}
\expandafter\def\csname PY@tok@gu\endcsname{\let\PY@bf=\textbf\def\PY@tc##1{\textcolor[rgb]{0.50,0.00,0.50}{##1}}}
\expandafter\def\csname PY@tok@ne\endcsname{\let\PY@bf=\textbf\def\PY@tc##1{\textcolor[rgb]{0.82,0.25,0.23}{##1}}}
\expandafter\def\csname PY@tok@sb\endcsname{\def\PY@tc##1{\textcolor[rgb]{0.73,0.13,0.13}{##1}}}
\expandafter\def\csname PY@tok@cm\endcsname{\let\PY@it=\textit\def\PY@tc##1{\textcolor[rgb]{0.25,0.50,0.50}{##1}}}
\expandafter\def\csname PY@tok@nd\endcsname{\def\PY@tc##1{\textcolor[rgb]{0.67,0.13,1.00}{##1}}}
\expandafter\def\csname PY@tok@nf\endcsname{\def\PY@tc##1{\textcolor[rgb]{0.00,0.00,1.00}{##1}}}
\expandafter\def\csname PY@tok@kc\endcsname{\let\PY@bf=\textbf\def\PY@tc##1{\textcolor[rgb]{0.00,0.50,0.00}{##1}}}
\expandafter\def\csname PY@tok@gs\endcsname{\let\PY@bf=\textbf}
\expandafter\def\csname PY@tok@mb\endcsname{\def\PY@tc##1{\textcolor[rgb]{0.40,0.40,0.40}{##1}}}
\expandafter\def\csname PY@tok@sx\endcsname{\def\PY@tc##1{\textcolor[rgb]{0.00,0.50,0.00}{##1}}}
\expandafter\def\csname PY@tok@dl\endcsname{\def\PY@tc##1{\textcolor[rgb]{0.73,0.13,0.13}{##1}}}
\expandafter\def\csname PY@tok@sc\endcsname{\def\PY@tc##1{\textcolor[rgb]{0.73,0.13,0.13}{##1}}}
\expandafter\def\csname PY@tok@vi\endcsname{\def\PY@tc##1{\textcolor[rgb]{0.10,0.09,0.49}{##1}}}
\expandafter\def\csname PY@tok@cs\endcsname{\let\PY@it=\textit\def\PY@tc##1{\textcolor[rgb]{0.25,0.50,0.50}{##1}}}
\expandafter\def\csname PY@tok@nc\endcsname{\let\PY@bf=\textbf\def\PY@tc##1{\textcolor[rgb]{0.00,0.00,1.00}{##1}}}
\expandafter\def\csname PY@tok@il\endcsname{\def\PY@tc##1{\textcolor[rgb]{0.40,0.40,0.40}{##1}}}
\expandafter\def\csname PY@tok@vm\endcsname{\def\PY@tc##1{\textcolor[rgb]{0.10,0.09,0.49}{##1}}}
\expandafter\def\csname PY@tok@ge\endcsname{\let\PY@it=\textit}
\expandafter\def\csname PY@tok@mi\endcsname{\def\PY@tc##1{\textcolor[rgb]{0.40,0.40,0.40}{##1}}}
\expandafter\def\csname PY@tok@o\endcsname{\def\PY@tc##1{\textcolor[rgb]{0.40,0.40,0.40}{##1}}}
\expandafter\def\csname PY@tok@sr\endcsname{\def\PY@tc##1{\textcolor[rgb]{0.73,0.40,0.53}{##1}}}
\expandafter\def\csname PY@tok@gh\endcsname{\let\PY@bf=\textbf\def\PY@tc##1{\textcolor[rgb]{0.00,0.00,0.50}{##1}}}
\expandafter\def\csname PY@tok@ss\endcsname{\def\PY@tc##1{\textcolor[rgb]{0.10,0.09,0.49}{##1}}}
\expandafter\def\csname PY@tok@nn\endcsname{\let\PY@bf=\textbf\def\PY@tc##1{\textcolor[rgb]{0.00,0.00,1.00}{##1}}}
\expandafter\def\csname PY@tok@gi\endcsname{\def\PY@tc##1{\textcolor[rgb]{0.00,0.63,0.00}{##1}}}
\expandafter\def\csname PY@tok@gp\endcsname{\let\PY@bf=\textbf\def\PY@tc##1{\textcolor[rgb]{0.00,0.00,0.50}{##1}}}
\expandafter\def\csname PY@tok@gd\endcsname{\def\PY@tc##1{\textcolor[rgb]{0.63,0.00,0.00}{##1}}}
\expandafter\def\csname PY@tok@cp\endcsname{\def\PY@tc##1{\textcolor[rgb]{0.74,0.48,0.00}{##1}}}
\expandafter\def\csname PY@tok@kr\endcsname{\let\PY@bf=\textbf\def\PY@tc##1{\textcolor[rgb]{0.00,0.50,0.00}{##1}}}
\expandafter\def\csname PY@tok@sd\endcsname{\let\PY@it=\textit\def\PY@tc##1{\textcolor[rgb]{0.73,0.13,0.13}{##1}}}
\expandafter\def\csname PY@tok@m\endcsname{\def\PY@tc##1{\textcolor[rgb]{0.40,0.40,0.40}{##1}}}
\expandafter\def\csname PY@tok@nb\endcsname{\def\PY@tc##1{\textcolor[rgb]{0.00,0.50,0.00}{##1}}}
\expandafter\def\csname PY@tok@cpf\endcsname{\let\PY@it=\textit\def\PY@tc##1{\textcolor[rgb]{0.25,0.50,0.50}{##1}}}
\expandafter\def\csname PY@tok@bp\endcsname{\def\PY@tc##1{\textcolor[rgb]{0.00,0.50,0.00}{##1}}}
\expandafter\def\csname PY@tok@mo\endcsname{\def\PY@tc##1{\textcolor[rgb]{0.40,0.40,0.40}{##1}}}
\expandafter\def\csname PY@tok@gr\endcsname{\def\PY@tc##1{\textcolor[rgb]{1.00,0.00,0.00}{##1}}}
\expandafter\def\csname PY@tok@ch\endcsname{\let\PY@it=\textit\def\PY@tc##1{\textcolor[rgb]{0.25,0.50,0.50}{##1}}}
\expandafter\def\csname PY@tok@sh\endcsname{\def\PY@tc##1{\textcolor[rgb]{0.73,0.13,0.13}{##1}}}
\expandafter\def\csname PY@tok@c\endcsname{\let\PY@it=\textit\def\PY@tc##1{\textcolor[rgb]{0.25,0.50,0.50}{##1}}}
\expandafter\def\csname PY@tok@na\endcsname{\def\PY@tc##1{\textcolor[rgb]{0.49,0.56,0.16}{##1}}}
\expandafter\def\csname PY@tok@kn\endcsname{\let\PY@bf=\textbf\def\PY@tc##1{\textcolor[rgb]{0.00,0.50,0.00}{##1}}}
\expandafter\def\csname PY@tok@err\endcsname{\def\PY@bc##1{\setlength{\fboxsep}{0pt}\fcolorbox[rgb]{1.00,0.00,0.00}{1,1,1}{\strut ##1}}}
\expandafter\def\csname PY@tok@ni\endcsname{\let\PY@bf=\textbf\def\PY@tc##1{\textcolor[rgb]{0.60,0.60,0.60}{##1}}}
\expandafter\def\csname PY@tok@no\endcsname{\def\PY@tc##1{\textcolor[rgb]{0.53,0.00,0.00}{##1}}}
\expandafter\def\csname PY@tok@kd\endcsname{\let\PY@bf=\textbf\def\PY@tc##1{\textcolor[rgb]{0.00,0.50,0.00}{##1}}}
\expandafter\def\csname PY@tok@vc\endcsname{\def\PY@tc##1{\textcolor[rgb]{0.10,0.09,0.49}{##1}}}
\expandafter\def\csname PY@tok@si\endcsname{\let\PY@bf=\textbf\def\PY@tc##1{\textcolor[rgb]{0.73,0.40,0.53}{##1}}}
\expandafter\def\csname PY@tok@nl\endcsname{\def\PY@tc##1{\textcolor[rgb]{0.63,0.63,0.00}{##1}}}
\expandafter\def\csname PY@tok@k\endcsname{\let\PY@bf=\textbf\def\PY@tc##1{\textcolor[rgb]{0.00,0.50,0.00}{##1}}}

\def\PYZbs{\char`\\}
\def\PYZus{\char`\_}
\def\PYZob{\char`\{}
\def\PYZcb{\char`\}}
\def\PYZca{\char`\^}
\def\PYZam{\char`\&}
\def\PYZlt{\char`\<}
\def\PYZgt{\char`\>}
\def\PYZsh{\char`\#}
\def\PYZpc{\char`\%}
\def\PYZdl{\char`\$}
\def\PYZhy{\char`\-}
\def\PYZsq{\char`\'}
\def\PYZdq{\char`\"}
\def\PYZti{\char`\~}
% for compatibility with earlier versions
\def\PYZat{@}
\def\PYZlb{[}
\def\PYZrb{]}
\makeatother


    % Exact colors from NB
    \definecolor{incolor}{rgb}{0.0, 0.0, 0.5}
    \definecolor{outcolor}{rgb}{0.545, 0.0, 0.0}



    
    % Prevent overflowing lines due to hard-to-break entities
    \sloppy 
    % Setup hyperref package
    \hypersetup{
      breaklinks=true,  % so long urls are correctly broken across lines
      colorlinks=true,
      urlcolor=urlcolor,
      linkcolor=linkcolor,
      citecolor=citecolor,
      }
    % Slightly bigger margins than the latex defaults
    
    \geometry{verbose,tmargin=1in,bmargin=1in,lmargin=1in,rmargin=1in}
    
    

    \begin{document}
    
    
    \maketitle
    
    

    
    \section{YAML Dateien}\label{yaml-dateien}

\begin{itemize}
\tightlist
\item
  Die erste Zeile eines Playbooks sollte mit "-\/-\/-" beginnen (drei
  Bindestriche). Diese zeigt den Beginn eines YAML-Dokumentes an.
\item
  Listen in YAML werden mit einem Bindestrich "-" gefolgt von einem
  Leerraum dargestellt.
\item
  Ein Playbook enthält eine Liste von Spielanweisungen; Sie werden mit
  "-" dargestellt. Jedes Spiel ist ein assoziatives Array, ein Dictonary
  oder eine Map in Form von Schlüssel/Wert-Paaren.
\item
  Einrückungen sind wichtig. Alle Mitglieder einer Liste sollten gleich
  Eingerückt sein.
\item
  Jede Spielanwiesung kann Schlüssel-Wert-Paare enthalten, getrennt
  durch ":", um Hosts, Variablen, Rollen, Aufgaben und so weiter
  anzugeben.
\end{itemize}

\subsection{Weblinks}\label{weblinks}

\begin{itemize}
\tightlist
\item
  https://de.wikipedia.org/wiki/YAML
\item
  http://www.yaml.org/
\item
  http://docs.ansible.com/ansible/YAMLSyntax.html
\end{itemize}

\subsection{Playbooks}\label{playbooks}

simple\_playbook.yml

\begin{Shaded}
\begin{Highlighting}[]
\OtherTok{---}
\KeywordTok{-} \FunctionTok{hosts:} \NormalTok{all}
  \FunctionTok{remote_user:} \NormalTok{vagrant}
  \FunctionTok{become:} \NormalTok{yes}
  \FunctionTok{tasks:}
  \KeywordTok{-} \FunctionTok{group:} \NormalTok{name=devops state=present}
  \KeywordTok{-} \FunctionTok{name:} \NormalTok{create devops user with admin previleges}
    \FunctionTok{user:} \NormalTok{name=devops comment="Devops User" uid=2001 group=devops}
  \KeywordTok{-} \FunctionTok{name:} \NormalTok{install htop package}
    \FunctionTok{apt:} \NormalTok{name=htop state=present update_cache=yes}

\KeywordTok{-} \FunctionTok{hosts:} \NormalTok{www}
  \FunctionTok{user:} \NormalTok{vagrant}
  \FunctionTok{become:} \NormalTok{yes}
  \FunctionTok{tasks:}
  \KeywordTok{-} \FunctionTok{name:} \NormalTok{add official nginx repository}
    \FunctionTok{apt_repository:} \NormalTok{repo='ppa:nginx/stable'}
  \KeywordTok{-} \FunctionTok{name:} \NormalTok{install nginx web server and ensure its at the latest version}
    \FunctionTok{apt:} \NormalTok{name=nginx state=latest}
  \KeywordTok{-} \FunctionTok{name:}
    \FunctionTok{service:} \NormalTok{name=nginx state=started}
\end{Highlighting}
\end{Shaded}

Alternativ

\begin{Shaded}
\begin{Highlighting}[]
\OtherTok{---}
\KeywordTok{-} \FunctionTok{hosts:} \NormalTok{all}
  \FunctionTok{remote_user:} \NormalTok{vagrant}
  \FunctionTok{become:} \NormalTok{yes}
  \FunctionTok{tasks:}
  \KeywordTok{-} \FunctionTok{group:}
      \FunctionTok{name:} \NormalTok{devops}
      \FunctionTok{state:} \NormalTok{present}
  \KeywordTok{-} \FunctionTok{name:} \NormalTok{create devops user with admin previleges}
    \FunctionTok{user:}
      \FunctionTok{name:} \NormalTok{devops}
      \FunctionTok{comment:} \StringTok{"Devops User"}
      \FunctionTok{uid:} \NormalTok{2001}
      \FunctionTok{group:} \NormalTok{devops}
  \KeywordTok{-} \FunctionTok{name:} \NormalTok{install htop package}
    \FunctionTok{apt:}
      \FunctionTok{name:} \NormalTok{htop}
      \FunctionTok{state:} \NormalTok{present}
      \FunctionTok{update_cache:} \NormalTok{yes}

\KeywordTok{-} \FunctionTok{hosts:} \NormalTok{www}
  \FunctionTok{user:} \NormalTok{vagrant}
  \FunctionTok{become:} \NormalTok{yes}
  \FunctionTok{tasks:}
  \KeywordTok{-} \FunctionTok{name:} \NormalTok{add official nginx repository}
    \FunctionTok{apt_repository:}
      \FunctionTok{repo:} \StringTok{'ppa:nginx/stable'}
  \KeywordTok{-} \FunctionTok{name:} \NormalTok{install nginx web server and ensure its at the latest version}
    \FunctionTok{apt:}
      \FunctionTok{name:} \NormalTok{nginx}
      \FunctionTok{state:} \NormalTok{latest}
  \KeywordTok{-} \FunctionTok{name:}
    \FunctionTok{service:}
      \FunctionTok{name:} \NormalTok{nginx}
      \FunctionTok{state:} \NormalTok{started}
\end{Highlighting}
\end{Shaded}

    \begin{Verbatim}[commandchars=\\\{\}]
{\color{incolor}In [{\color{incolor}1}]:} cp /vagrant/simple\PYZus{}playbook.yml ./
\end{Verbatim}


    \begin{Verbatim}[commandchars=\\\{\}]

    \end{Verbatim}

    \begin{Verbatim}[commandchars=\\\{\}]
{\color{incolor}In [{\color{incolor}2}]:} \PY{n+nb}{eval} \PY{l+s+sb}{`}ssh\PYZhy{}agent \PYZhy{}s\PY{l+s+sb}{`} \PYZgt{} /dev/null
        ./ssh\PYZhy{}add\PYZhy{}passphrase.sh
\end{Verbatim}


    \begin{Verbatim}[commandchars=\\\{\}]

Enter passphrase for /home/vagrant/.ssh/id\_rsa: 
Identity added: /home/vagrant/.ssh/id\_rsa (/home/vagrant/.ssh/id\_rsa)

    \end{Verbatim}

    \begin{Verbatim}[commandchars=\\\{\}]
{\color{incolor}In [{\color{incolor}3}]:} ansible\PYZhy{}playbook simple\PYZus{}playbook.yml
\end{Verbatim}


    \begin{Verbatim}[commandchars=\\\{\}]

PLAY [all] *********************************************************************

TASK [setup] *******************************************************************
ok: [192.168.60.21]
ok: [192.168.60.11]
ok: [192.168.60.2]
ok: [192.168.60.12]
ok: [192.168.60.22]
ok: [localhost]
ok: [192.168.60.13]

TASK [group] *******************************************************************
changed: [192.168.60.21]
changed: [192.168.60.2]
changed: [192.168.60.12]
changed: [192.168.60.22]
changed: [192.168.60.11]
changed: [localhost]
changed: [192.168.60.13]

TASK [create devops user with admin previleges] ********************************
changed: [192.168.60.22]
changed: [192.168.60.11]
changed: [192.168.60.12]
changed: [192.168.60.2]
changed: [192.168.60.21]
changed: [192.168.60.13]
changed: [localhost]

TASK [install htop package] ****************************************************
ok: [192.168.60.22]
ok: [192.168.60.2]
ok: [192.168.60.11]
ok: [192.168.60.12]
ok: [192.168.60.21]
ok: [192.168.60.13]
ok: [localhost]

PLAY [www] *********************************************************************

TASK [setup] *******************************************************************
ok: [192.168.60.11]
ok: [192.168.60.13]
ok: [192.168.60.12]

TASK [add official nginx repository] *******************************************
ok: [192.168.60.13]
ok: [192.168.60.12]
ok: [192.168.60.11]

TASK [install nginx web server and ensure its at the latest version] ***********
changed: [192.168.60.11]
changed: [192.168.60.12]
changed: [192.168.60.13]

TASK [service] *****************************************************************
ok: [192.168.60.11]
ok: [192.168.60.13]
ok: [192.168.60.12]

PLAY RECAP *********************************************************************
192.168.60.11              : ok=8    changed=3    unreachable=0    failed=0   
192.168.60.12              : ok=8    changed=3    unreachable=0    failed=0   
192.168.60.13              : ok=8    changed=3    unreachable=0    failed=0   
192.168.60.2               : ok=4    changed=2    unreachable=0    failed=0   
192.168.60.21              : ok=4    changed=2    unreachable=0    failed=0   
192.168.60.22              : ok=4    changed=2    unreachable=0    failed=0   
localhost                  : ok=4    changed=2    unreachable=0    failed=0   


    \end{Verbatim}

    \subsection{Playbooks}\label{playbooks}

Wichtige Abschnitte der obigen Playbooks sind:

\begin{enumerate}
\def\labelenumi{\arabic{enumi}.}
\tightlist
\item
  Wer soll wie konfiguriert werden (hosts).
\item
  Was soll ablaufen (tasks).
\end{enumerate}

\subsection{Pattern für hosts}\label{pattern-fuxfcr-hosts}

Im vorherigen Playbook bestimmen die folgenden Zeilen, welche Hosts für
einen Play/Spiel ausgewählt werden sollen. Eine bestimmte Spielanweisung
für:

\begin{itemize}
\tightlist
\item
  hosts: all
\item
  hosts: www
\end{itemize}

Der erste Block wird mit allen Hosts ausgeführt. Der zweite
Abschnitt/Play wird mit der www-Gruppe durchgeführt.

Die Pattern können eine der folgenden Liste, oder ihre Kobinationen,
sein:

\begin{verbatim}
Pattern               Beispiele
Gruppenname           Name_der_Rechner (ansible inventory)
Spiel alle            all oder *
Range                 Name_des_Rechner[0:100]
Hostnamen globs       *.example.com, host01.example.com
Ausnahmen             Name_der_Rechner:!diesen_nicht
Überschneidung        Name_der_Rechner:&weitere_Namen
Reguläre Ausdrücke    ~(nn|zk).*\.example\.org
\end{verbatim}

\subsection{Der Block Tasks}\label{der-block-tasks}

Die Aufgaben für eine Gruppe (hosts). Aufgaben sind eine Folge von
Aktionen, die gegen eine Gruppe von Hosts ausgeführt werden. Jedes Play
enthält in der Regel mehrere Tasks, die seriell auf jeder Maschine
ausgeführt werden, die dem Muster entspricht.

Jede Aktion in einer Aufgabenliste kann deklariert werden, indem
Folgendes angegeben wird:

\begin{itemize}
\tightlist
\item
  Der Name des Moduls
\item
  Optional der Zustand der verwalteten Systemkomponente
\item
  Die optionalen Parameter
\end{itemize}

\subsection{Module}\label{module}

Module sind die gekapselten Prozeduren, die spezifischen
Systemkomponenten für bestimmten Plattformen verwalten z. B. apt, user
oder service.

http://docs.ansible.com/ansible/list\_of\_all\_modules.html

Nichts gefunden? Dann selber schreiben:

http://docs.ansible.com/ansible/developing\_modules.html

\subsection{Module und Idempotence}\label{module-und-idempotence}

Aus der Wikipedia (10/2016) "Analog dazu wird in der Informatik ein
Stück Programmcode, das mehrfach hintereinander ausgeführt das gleiche
Ergebnis wie bei einer einzigen Ausführung liefert, als idempotent
bezeichnet."

    \begin{Verbatim}[commandchars=\\\{\}]
{\color{incolor}In [{\color{incolor}4}]:} ansible\PYZhy{}playbook simple\PYZus{}playbook.yml
\end{Verbatim}


    \begin{Verbatim}[commandchars=\\\{\}]

PLAY [all] *********************************************************************

TASK [setup] *******************************************************************
ok: [192.168.60.12]
ok: [192.168.60.11]
ok: [192.168.60.21]
ok: [192.168.60.2]
ok: [192.168.60.22]
ok: [192.168.60.13]
ok: [localhost]

TASK [group] *******************************************************************
ok: [192.168.60.2]
ok: [192.168.60.12]
ok: [192.168.60.21]
ok: [192.168.60.22]
ok: [192.168.60.11]
ok: [localhost]
ok: [192.168.60.13]

TASK [create devops user with admin previleges] ********************************
ok: [192.168.60.21]
ok: [192.168.60.2]
ok: [192.168.60.12]
ok: [192.168.60.22]
ok: [192.168.60.11]
ok: [192.168.60.13]
ok: [localhost]

TASK [install htop package] ****************************************************
ok: [192.168.60.21]
ok: [192.168.60.2]
ok: [192.168.60.11]
ok: [192.168.60.12]
ok: [192.168.60.22]
ok: [localhost]
ok: [192.168.60.13]

PLAY [www] *********************************************************************

TASK [setup] *******************************************************************
ok: [192.168.60.13]
ok: [192.168.60.11]
ok: [192.168.60.12]

TASK [add official nginx repository] *******************************************
ok: [192.168.60.11]
ok: [192.168.60.13]
ok: [192.168.60.12]

TASK [install nginx web server and ensure its at the latest version] ***********
ok: [192.168.60.12]
ok: [192.168.60.11]
ok: [192.168.60.13]

TASK [service] *****************************************************************
ok: [192.168.60.11]
ok: [192.168.60.12]
ok: [192.168.60.13]

PLAY RECAP *********************************************************************
192.168.60.11              : ok=8    changed=0    unreachable=0    failed=0   
192.168.60.12              : ok=8    changed=0    unreachable=0    failed=0   
192.168.60.13              : ok=8    changed=0    unreachable=0    failed=0   
192.168.60.2               : ok=4    changed=0    unreachable=0    failed=0   
192.168.60.21              : ok=4    changed=0    unreachable=0    failed=0   
192.168.60.22              : ok=4    changed=0    unreachable=0    failed=0   
localhost                  : ok=4    changed=0    unreachable=0    failed=0   


    \end{Verbatim}


    % Add a bibliography block to the postdoc
    
    
    
    \end{document}
